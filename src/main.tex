\documentclass[11p]{article}
% Packages
\usepackage{amsmath}
\usepackage{graphicx}
\usepackage[swedish]{babel}
\usepackage[
    backend=biber,
    style=authoryear-ibid,
    sorting=ynt
]{biblatex}
\usepackage[utf8]{inputenc}
\usepackage[T1]{fontenc}
%Källor
\addbibresource{mall.bib}
\graphicspath{ {./images/} }

\title{PMmall \\ \small Fysik 1}
\author{Magnus Silverdal }
\date{\today}

\begin{document}

    \begin{titlepage}
        \begin{center}
            \vspace*{1cm}

            \Huge
            \textbf{PM Vattenkraftverk}

            \vspace{0.5cm}
            \LARGE


            \vspace{1.5cm}

            \textbf{Alfred Engberg}

            \vfill

            Ett PM om energiförsörjning \\
            Fysik 1

            \vspace{0.8cm}


            \Large
            Teknikprogrammet\\
            NTI Gymnasiet\\
            Umeå\\
            \today

        \end{center}
    \end{titlepage}
% Om arbetet är långt har det en innehållsförteckning, annars kan den utelämnas
    \tableofcontents
    \newpage

    \section{Inledning}
    Vattenkraft är en av dom tre primära sätten att tillverka elektricitet på ett förnybart sätt.
    Där dom andra är Sol och vind.
    Syftet med detta PM är att lyfta fram miljöpåverkan av vattenkraftverk och ta fram för- och nackdelar för miljön och samhället.

    \subsection{frågeställningar}
    I detta PM kommer följande frågor besvaras:
    \begin{enumerate}
        \item Hur fungerar ett vattenkraftverk?
        \item Hur påverkar vattenkraftverk miljön?
        \item Hur påverkar vattenkraftverk samhället?
    \end{enumerate}

    \secton{Resultat}

    \subsection{Så fungerar ett vattenkraftverk}
    Ett vattenkraft fungerar genom att man dammar av en bit av ett vattendrag till exempel en älv.
    Den bit som är avdammad kallas för ett vattenmagasin som förvarar vatten.
    Det lagrade vattnet hålls in av dammen, vattnet har energi i form av läges energi.
    När man vill utvinna elektricitet öppnas luckor i dammen som tillåter vatten att falla ner i tunnlar till en turbin.
    Vattnet snurrar turbinen som skapar rörelse energi som i sin tur driver en generator som skapar elektriciteten.
    Mängden elektricitet man får ut varierar på mängden vatten, ju mer vatten man har i vattenmagasinet ju högre vattentryck och det innebär att turbinen kan snurra snabbare.
    \parencite{vattenfall}

    \subsection{Globala miljökonsekvenser av vattenkraftverk}
    Globalt har vattenkraftverk näst intill ingen miljöpåverkan.
    Den ända miljöpåverkan som vattenkraftverk har är själva konstruktionen av det och reservdelar till det.
    Men annat än det så är det inget dåligt, speciellt om man jämför med fossila energikällor som till exempel kol.
    \parencite{FornybaraEnergikAllorsInverkanPaDeSvenskaMiljomalen}

    \subsection{Lokal miljöpåverkan av ett vattenkraftverk}
    Den lokala miljön omkring vattenkraftverket kommer att förstöras.
    Med hur ett vattenkraftverk fungerar kommer vattenmagasinet dela upp älvar i olika delar.
    \parencite{VattenkraftForOchEmot}


    Delar av älven uppströms kommer att dränkas och ändra hur det lokala ekosystemet ser ut där.
    Dammarna skapar också en barrier för fiskar som exempel lax som migrerar uppströms för att fotplanta sig.
    \parencite{FornybaraEnergikAllorsInverkanPaDeSvenskaMiljomalen}

    \subsection{Hur påverkar vattenkraftverk samhället?}
    Vattenkraftverk består av dammar som kommer att kapa av älvar, och därmed också transport längs vatten i båtar.
    Det gör då att om man äger en båt så kan man inte åka runt i älven hur man vill, utan man begränsas av dammen.
    Men det kommer att skapa jobb att bygga dammen och sen att se till att den fungerar.
    Ett mindre problem med vattenkraftverk är att dom kan förstöra utsikter för vissa människor.

    Den primära fördelen med ett vattenkraftverk för samhället är just att det är ett kraftverk.
    Vattenkraftverk producerar billig och pålitlig elektricitet.

    Försvaret kan använda älvar som en barrier för teoretiska fiender.
    Med dammar kommer älvarna nedströms om dammen vara betydligt mindre.
    \parencite{VattenkraftForOchEmot}

    \section{Slutsatser}
    för att sammanfatta kan man säga på en global skala har vattenkraftverk bara en bra påverkan.
    Man bygger dom en gång och sen så får man praktiskt sätt gratis el.
    Medans till exempel kolkraftverk kräver kol för att skapa elektricitet.
    Men den lokala påverkan är dock större.
    Dammar ändrar flödet av vatten och förhindrar vandring av fisk och djur på land.
    Vattenkraftverk skapar billig och pålitlig el.
    För samhället så är det överlag en bra sak att bygga dammar och vattenkraftverk.

    \printbibliography

\end{document}

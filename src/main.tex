\documentclass[11p]{article}
% Packages
\usepackage{amsmath}
\usepackage{graphicx}
\usepackage[swedish]{babel}
\usepackage[
    backend=biber,
    style=authoryear-ibid,
    sorting=ynt
]{biblatex}
\usepackage[utf8]{inputenc}
\usepackage[T1]{fontenc}
%Källor
\addbibresource{mall.bib}
\graphicspath{ {./images/} }

\title{PMmall \\ \small Fysik 1}
\author{Magnus Silverdal }
\date{\today}

\begin{document}

    \begin{titlepage}
        \begin{center}
            \vspace*{1cm}

            \Huge
            \textbf{Title}

            \vspace{0.5cm}
            \LARGE
            Subtitle

            \vspace{1.5cm}

            \textbf{Alfred Engberg}

            \vfill

            Ett PM om energiförsörjning \\
            Fysik 1

            \vspace{0.8cm}


            \Large
            Teknikprogrammet\\
            NTI Gymnasiet\\
            Umeå\\
            \today

        \end{center}
    \end{titlepage}
% Om arbetet är långt har det en innehållsförteckning, annars kan den utelämnas
    \tableofcontents
    \newpage
    \section{Disposition hos ett PM}
    Ett PM har den mest informella strukturen av de vetenskapliga texterna.
    Det är egentligen bara en sammanställning av kunskap men för att den ska bli lite lättare att ta sig an brukar det finnas en inledning där syfte och frågeställningar redovisas och en avslutning där du kan dra slutsatser.
    Rubrikerna kan döpas valfritt, speciellt de som finns i huvuddelen av texten beror på vad den handlar om.
    Se nedan för ett exempel.

    \section{Inledning}
    Beskriv varför detta ämne är intressant eller viktigt. Vad är syftet med texten?
    \subsection{frågeställningar}
    I detta PM kommer följande frågor besvaras:
    \begin{enumerate}
        \item Hur fungerar ett vattenkraftverk?
        \item Hur påverkar vattenkraftverk miljön?
        \item Hur påverkar vattenkraftverk smmhället?
    \end{enumerate}

    \section{Resultat}
    Här kommer allt med massor av mer rubriker och underrubriker

    \subsection{Vattenkraftverk, så fungerar det}
    Ett vattenkraft fungerar genom att man dammar av en bit av ett vattendrag till exempel en älv.
    Den bit som är avdammad kallas för ett vattenmagasin som förvarar vatten.
    Det lagrade vattnet hålls in av dammen, vattnet har energi i form av läges energi.
    När man vill utvinna elektricitet öppnas luckor i dammen som tillåter vatten att falla ner i tunnlar till en turbin.
    Vattnet snurrar turbinen som skapar rörelse energi som i sin tur driver en generator som skapar elektriciteten.
    Mängden electricitet man får ut varierar på mängden vatten, ju mer vatten man har i vattenmagasinet ju högre vattentryck och det innebär att turbinen kan snurra snabbare.
    \parencite{vattenfall}

    \subsection{Globala miljökonsekvenser av vattenkraftverk}
    gLOBAT INA utsläuu :)

    \subsection{Lokal miljöpåverkan av ett vattenkraftverk}
    Den lokala miljön omkring dammen kommer att förstöras både av byggandet och av dammen.
    \parencite{Vattenkraft: för och mot}


    \subsection{Solkraft bidrar till att minska konflikter om oljetillgångar i världen}


    \subsection{}

    \section{Slutsatser}
    Här kan du dra slutsatser eller sammanfatta ditt resultat

% Mer saker som du kan ha nytta av.

    \section{Referenser}
    Referenser i text kan skrivas på två sätt: Enligt \textcite{Jens} kan man använde två typer av referenser, inbäddade i texten eller efter ett fakta \parencite{Fraenkel}. Ett till test för att se hur det ser ut \parencite[sid 55]{fermi}.

    \section{Annat som kan vara bra att veta}
    Om du vill ha kodstil och få med alla tecken kan du använda verbatim. då kan du skriva \verb|abcd!"#| utan problem...

    Citat skrivs mellan de konstiga symbolerna \verb|``| och \verb|''| för att de ska se bra ut ``se bra ut!''.
    \subsection{En underrubrik}
    \subsubsection{En underunderrubrik}
    \subsection{Ekvationer}
    Det är lätt att skriva matematik i \LaTeX

    \begin{equation}
        F = G \frac{M m}{r^2}
        \label{grav}
    \end{equation}

    Ekvation (\ref{grav}) känner ni igen...

    \subsection{figurer}
    Bilder placeras enklast på detta sätt. placeringen bestämmer \LaTeX och vi kan bara föreslå (h)är, (t)opp eller (b)otten. Ett utropstecken före tvingar lite mer men inte absolut. I bild \ref{varg} visas en varg
    \begin{figure}[!h]
        \caption{Acceleration-tid diagram. Källa: Impuls Fysik 1}
        \label{varg}
    \end{figure}
    \printbibliography

\end{document}
